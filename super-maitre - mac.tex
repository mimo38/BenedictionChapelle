
\documentclass[11.5pt,latin,french,a5paper,twoside]{book}
%\usepackage[top=1cm,bottom=1cm,inner=1.1cm,outer=1cm]{geometry}
\newcounter{facteur}
\setcounter{facteur}{12}
\usepackage{packages/gredoc}
%\pagestyle{empty}
\title{Heure Sainte}\date{}\author{ }
\rouge
\redlines

\title{\textsc{\Huge{Inauguration}}}
\date{20 décembre 2014}

%\addcontentsline{toc}{niveau}{titre}

\begin{document}


%\thispagestyle{empty}
\maketitle

%\setcounter{page}{1}
\pagebreak
\titre{Saint Anthelme\\Évêque de Belley}
\begin{center}
{\includegraphics[height=6cm]{images/medium_Antelme_eveque.jpg}}
\end{center}

\begin{quotation}
\thispagestyle{empty}
Né au château de Chignin, Anthelme devient prêtre en 1135. Attiré par l'exemple des Chartreux, il se joint à eux, au monastère de Portes, puis à la Grande-Chartreuse, où il est très vite élu septième prieur. Il se consacre alors à la direction du monastère avec zèle, alliant les affaires temporelles avec une vie spirituelle exemplaire. De ce zèle découla le développement des Chartreux en Europe, et l'apparition de la branche féminine de l'ordre.
Sa fermeté dans la défense du bien lui attire des ennuis et de fausses accusations auprès du Pape Eugène III, mais saint Bernard prend sa défense, et son innocence est reconnue. Regrettant alors la paix de la solitude, il démissione du priorat.

Mais à l'élection d'Alexandre III, un antipape, favori de l'empereur Frédéric Barberousse, trouble la paix de l'Église. Anthelme travaille alors à soutenir Alexandre III dans son droit, entraînant derrière lui toute l'Europe, malgré les intimidations de l'empereur.

 Sa renommée devient alors telle que le Pape le nomme bientôt évêque de Belley en 1163.
Il gouverne alors son diocèse avec charité et vigilance, réformant les m{\oe}urs de son clergé et de ses fidèles, sachant défendre les droits de l'Église contre les grands de ce monde et gardant toujours les yeux tournés vers la Chartreuse où il aime se retirer de temps en temps.
Il meurt le 26 juin 1178, empporté par la fièvre alors qu'il soulageait les misères des pauvres souffrants de la famine.
Dieu confirma la sainteté de son serviteur par des miracles le jour même de sa sépulture

Ce Saint nous offre une triple gloire : il a renoncé à tous les avantages temporels ; il a affermi un ordre naissant qui allait se multiplier ; il a puissament contribué à empêché un schisme dans l'Église.
\end{quotation} 
\part{Bénédiction}

%\addcontentsline{toc}{chapter}{A l'extérieur}
\rubrique{Le début de la cérémonie est à l'extérieur de l'église (ou de l'oratoire) que l'on va bénir. Précédé de la croix et du clergé, l'officiant s'arrête devant la porte fermée de l'église, et dit cette oraison :}
\traduire{Actiónes nostras, qu{\'\ae}sumus, Dómine, aspirándo pr{\'\ae}veni, et adjuvándo proséquere: ut cuncta nostra orátio et operátio a te semper incípiat, et per te c\oe pta finiátur. Per Christum Dóminum nostrum.}{Seigneur, devancez nos actes par votre inspiration, et soutenez-les par votre aide, afin que toutes nos prières et nos actions commencent toujours par vous et, une fois entreprises, par vous s'achèvent. Par le Christ Notre-Seigneur.}
\Amen
\rubrique{Ensuite, le prêtre entonne l'antienne que continue le ch\oe ur :}
\traduction{Vous m'arroserez avec l'hysope, Seigneur, et je serai purifié : Vous me laverez, et je deviendrai plus blanc que la neige.}
\partition{Antienne}{AspergesMeSolus}{7.}
\rubrique{Le prêtre asperge alors les murs extérieurs de l'église ; pendant ce temps, on chante le psaume \emph{Miserere}.}
\titred{Psaume 50}
\rubrique{L'\'Eglise des pécheurs exprime le regret de ses enfants contrits et implore le pardon du Christ, source de joie.}
\partition{Psaume}{Intonation-050-7a}{}
\medskip
\psaume{Ps-050-7}
%\psalmus[%
%tonus=4,%
%primus=2,numerus=2,%
%ultimus=10,%
%]{050}
%\gloria[tonus=7]
%\rubrique{}
%\traduire{
%\input{../gabc/Psaume/Ps-050brut}
%}
%{
%\input{\input{../gabc/Psaume/Ps-050brut}}
%}
\partition{Antienne}{AspergesMeSolus}{7.}

\flectamus
\traduire{Dómine Deus, qui licet c\ae lo et terra non capiáris, domum tamen dignáris habére in terris, ubi nomen tuum júgiter invocétur : locum hunc qu{\'\ae} umus, beátæ Maríæ semper Vírginis, et beáti \emph{Anthélmi}, omniúmque Sanctórum intercedéntibus méritis, seréno pitetátis tuæ intúitu vísita, et per infusiónem grátiæ tuæ ab omni inquiaménto purífica, purificatúmque consérva ; et qui dilécti tui David devotiónem in fílii sui Salomónis ópere complevisti, in hoc ópere desidéria nostra perfícere dignéris, effugiéntque omnes hinc nequítiæ spirituáles. \Perdominum  \quitecum \peromnia}{
Seigneur Dieu, vous qui ne désirez ni le ciel ni la terre, vous daignez avoir une habitation terrestre, où votre nom sera invoqué, daignez s'il vous plaît visiter ce lieu dans votre bienveillante piété, par les mérites de la bienheureuse Vierge Marie, de saint \emph{Anthelme} et de l'intercession de tous les saints ; purifiez-le de toute souillure, par l'infusion de votre grâce, et gardez le sans tâches ; et vous, qui avez complété la piété de David par l'{\oe}uvre de Salomon, daignez parfaire nos désirs dans cettte {\oe}uvre, et que tout mal spirituel s'éloigne de ce lieu. Par Notre-Seigneur Jésus-Christ votre Fils, qui, étant Dieu, vit et règne avec vous, en l'unité du même Saint-Esprit, dans tous les siècles des siècles.
}
\Amen
\rubrique{La procession entre alors dans l'église. On chante ensuite les litanies à genoux :}
\titre{Litanies des Saints}
%\partition{Litanies}{kyrie}{}
%\partition{Litanies}{propitius}{}
%\partition{Litanies}{peccatores}{}
%\partition{Litanies}{conclusion}{}
\includescore{../gabc/litanies/litanies.tex}

\rubrique{Le prêtre se lève}
\flectamus
\traduire{Prævéniat nos, qu{\'æ}sumus, Dómine, misericórdia tua : et, intercedéntbus ómnibus Sanctis tuis, voces nostras cleméntia tuæ propitiatiónis antícpet. Per Christum Dóminum nostrum.}
{Nous vous en prions, Seigneur, que votre Miséricorde nous vienne en aide, et par l'intercession de tous vos saints, que le secours de votre bonté devance nos prières. Par le Christ Notre-Seigneur.}
\Amen
\rubrique{Le prêtre entonne :}
\partition{Verset}{DeusInAdiutoriumSineAlleluia}{\V}

\flectamus
\traduire{Omnípotens et miséricors Deus, qui Sacerdótibus tuis tantam præ céteris grátiam contulísti, ut quidquid in tuo nómine digne, perfectequé ab eis ágitur, a te fíeri credátur : qu\'æ sumus imménsam cleméntiam tuam; ut quidquid modo visitáturi sumus, vísites, et quidquid benedictúri sumus, benedícas : sitque ad nostræ humilitátis intróitum, Sanctórum tuórum méritis, fuga d\'æ monum, Angeli pacis ingréssus. \Perdominum \quitecum \peromnia}
{
Dieu Tout-Puissant et Miséricordieux, qui conférez à vos prêtres, parmi bien d'autres grâces, que tout ce qui est réalisé dignement par eux en l'honneur de votre nom, soit perfectionné par vous, comme nous le croyons ; nous implorons votre clémence infinie, pour que ce que nous visitons aujourd'hui reçoive également votre visite ; et pour que ce que nous bénissons soit béni par vous. Que, par les mérites de vos saints, notre humble procession entraîne la fuite des démons et la venue de l'Ange de paix. Par Notre-Seigneur Jésus-Christ votre Fils, qui, étant Dieu, vit et règne avec vous, en l'unité du même Saint-Esprit, dans tous les siècles des siècles.
}
\Amen
\rubrique{Le prêtre entonne alors l'antienne : }

\partition{Antienne}{BenedicDomine}{3a}

\rubrique{On chante alors les trois psaumes suivants. Pendant ce temps, le prêtre asperge les murs intérieurs en récitant l'antienne \emph{Asperges me.}}
\titred{Psaume 119}
\medskip
\rubrique{Héritière de la vérité et de la paix du Christ, l'Église prie son Seigneur de la sauver.}
\traduction{Dans ma tribulation j'ai crié vers le Seigneur et Il m'a exaucé.}
\partition{Psaume}{Intonation-119-3a}{}

\psaume{Ps-119-3a}

\titred{Psaume 120}
\rubrique{L'Église prie le Christ don Seigneur de veiller sur elle, en marche vers la cité éternelle.}
\psaume{Ps-120-3a}

\titred{Psaume 121}
\rubrique{L'Église chante sa joie d'être en marche derrière le Christ vers la Jérusalem des cieux.}
\psaume{Ps-121-3a}

\partition{Antienne}{BenedicDomine}{3a}

\rubrique{De retour à l'autel le prêtre dit alors :}
\flectamus
\traduire{Deus qui loca nómini tuo dicánda sanctíficas, effúnde super hanc oratiónis domum grátiam tuam: ut ab ómnibus hic nomen tuum invocántibus auxílium tuæ misericórdiæ sentiátur. \Perdominum \quitecum \peromnia}
{Dieu, qui sanctifiez les lieux qui seront consacrés à votre nom, répandez votre grâce sur cette maison de prière, afin que tous ceux qui invoquent ici votre nom éprouvent le secours de votre miséricorde. Par Notre-Seigneur Jésus-Christ votre Fils qui, étant Dieu, vit et règne avec vous, en l'unité du même Saint-Esprit, dans tous les siècles des siècles.}
\Amen

\rubrique{La bénédiction accomplie, on célèbre la messe du Mystère ou du Saint, en l'honneur de qui on a béni l'Église.}



\part{Messe de saint Anthelme}

\begin{center}\titreb{Messe des catéchumènes}\end{center}
\titre{Introït}
\traduction{Le Seigneur fit avec lui une alliance de paix et l'établit prince, afin que la dignité sacerdotale lui appartînt toujours. Souvenez-vous, Seigneur, de David et de toute sa douceur.}
\partition{Introit}{StatuitSineAlleluia}{4.}

\titre{Kyrie IV}
\traduction{Seigneur ayez pitié de nous. Jésus-Christ ayez pitié de nous. Seigneur ayez pitié de nous.}
\partition{Kyriale}{Kyrie-IV}{1.}
\titre{Gloria}
\traduction{Gloire à Dieu au plus haut des deux, et paix sur la terre aux hommes de bonne volonté. Nous vous louons, nous vous bénissons, nous vous adorons, nous vous glorifions et nous vous rendons grâces [8] pour votre gloire immense, Seigneur Dieu, Roi du ciel, Dieu Père tout-puissant. Seigneur Fils unique, Jésus-Christ, Seigneur Dieu, Agneau de Dieu, Fils du Père, vous qui enlevez les péchés du monde ayez pitié de nous, vous qui enlevez les péchés du monde [9] accueillez notre prière, vous qui siégez à la droite du Père, ayez pitié de nous. Car c'est vous le seul Saint, vous le seul Seigneur, vous le seul Très-Haut, Jésus-Christ, avec le Saint-Esprit, dans la gloire de Dieu le Père. Ainsi soit-il.}
\partition{Kyriale}{Gloria-IV}{4.}

\titre{Collecte}
\medskip
\traduire{\dominus}{\leseigneur}
\bigskip
\traduire{Deus, qui perénnem glóriam sanctissimi Confessoris tui atque Pontificis Anthélmi ánimæ contulísti : tríbue q\'æ sumus, ejus nos ita apud te patrocíniis sublimári, ut cum eo vitam possideámus ætérnam.\Perdominum \quitecum \peromnia }{Ô Dieu qui avez accordé une gloire inaltérable à votre saint Confesseur et Pontife Anthelme, accordez-nous, nous vous en prions, d'être tellement soutenus par son patronage, que nous puissions avec lui posséder la vie éternelle. \Parjesus \quietant \siecles}\nopagebreak[4]
\nopagebreak[4]\Amen

\titre{Épître}
\rubrique{Eccli. 44, 16-27 ; 45, 3-20.}
\traduire{
Ecce sacérdos magnus, qui in diébus suis plácuit Deo, et invéntus est iustus : et in témpore iracúndiæ factus est reconciliátio. Non est invéntus símilis illi, qui conservávit legem Excélsi. Ideo iureiurándo fecit illum Dóminus créscere in plebem suam. Benedictiónem ómnium géntium dedit illi, et testaméntum suum confirmávit super caput eius. Agnóvit eum in benedictiónibus suis : conservávit illi misericórdiam suam : et invenit grátiam coram óculis Dómini. Magnificávit eum in conspéctu regum : et dedit illi corónam glóriæ. Státuit illi testaméntum ætérnum, et dedit illi sacerdótium magnum : et beatificávit illum in glória. Fungi sacerdótio, et habére laudem in nómine ipsíus, et offérre illi incénsum dignum in odórem suavitátis.
}
{
Voici le grand pontife, qui pendant les jours de sa vie fut agréable à Dieu, et est devenu, au temps de sa colère, la réconciliation des hommes. Nul ne l'a égalé dans l'observation des lois du Très-Haut. C'est pourquoi le Seigneur a jure de le rendre père de son peuple. Le Seigneur a béni en lui toutes les nations et a confirmé en lui son alliance. Il a versé sur lui ses bénédictions ; il lui a continué sa miséricorde ; et cet homme a trouve grâce aux yeux du Seigneur. Celui-là l'a rendu grand devant les rois, et il lui a donné une couronne de gloire. Il a fait avec lui une alliance éternelle ; il lui a donné le suprême sacerdoce, et il l'a rendu heureux dans la gloire, pour exercer le sacerdoce, louer son nom et lui offrir dignement un encens d'agréable odeur.
}
\Deo

\titre{Graduel}
\traduction{Voici le grand Pontife qui dans les jours de sa vie a plu à Dieu. Nul ne lui a été trouvé semblable, lui qui a conservé la loi du Très-Haut.}
\partition{Graduel}{EcceSacerdos}{}

\titre{Alleluia}
\traduction{Allelúia, allelúia. Vous êtes prêtre à jamais selon l'ordre de Melchisédech. Alléluia.}
\partition{Alleluia}{TuEsSacerdos}{}

\titre{\'Evangile}
\rubrique{Matth. 25, 14-23.}
\traduire{
In illo témpore : Dixit Iesus discípulis suis parábolam hanc : Homo péregre proficíscens vocávit servos suos, et trádidit illis bona sua. Et uni dedit quinque talénta, álii a tem duo, álii vero unum, unicuíque secúndum própriam virtútem, et proféctus est statim. Abiit autem, qui quinque talénta accéperat, et operátus est in eis, et lucrátus est ália quinque. Simíliter et, qui duo accéperat, lucrátus est ália duo. Qui autem unum accéperat, ábiens fodit in terram, et abscóndit pecúniam dómini sui. Post multum vero témporis venit dóminus servórum illórum, et pósuit ratiónem cum eis. Et accédens qui quinque talénta accéperat, óbtulit ália quinque talénta,dicens : Dómine, quinque talénta tradidísti mihi, ecce, ália quinque superlucrátus sum. Ait illi dóminus eius : Euge, serve bone et fidélis, quia super pauca fuísti fidélis, super multa te constítuam : intra in gáudium dómini tui. Accéssit autem et qui duo talénta accéperat, et ait : Dómine, duo talénta tradidísti mihi, ecce, ália duo lucrátus sum. Ait illi dóminus eius : Euge, serve bone et fidélis, quia super pauca fuísti fidélis, super multa te constítuam : intra in gáudium dómini tui.
}
{
En ce temps-là, Jésus dit à ses disciples cette parabole : Un homme, partant pour un long voyage, appela ses serviteurs et leur remit ses biens. Il donna à l'un cinq talents, à un autre deux, et à un autre un seul, à chacun selon sa capacité ; puis il partit aussitôt. Celui qui avait reçu cinq talents s'en alla, les fit valoir, et en gagna cinq autres. De même, celui qui en avait reçu deux, en gagna deux autres. Mais celui qui n'en avait reçu qu'un, s'en alla, creusa dans la terre et cacha l'argent de son maître. Longtemps après, le maître de ces serviteurs revint, et leur fit rendre compte. Et celui qui avait reçu cinq talents s'approcha, et présenta cinq autres talents, en disant : Seigneur, vous m'avez remis cinq talents ; voici que j'en ai gagné cinq autres. Son maître lui dit : C'est bien, bon et fidèle serviteur ; parce que tu as été fidèle en peu de choses, je t'établirai sur beaucoup ; entre dans la joie de ton maître. Celui qui avait reçu deux talents s'approcha aussi, et dit : Seigneur, vous m'avez remis deux talents ; voici que j'en ai gagné deux autres. Son maître lui dit : C'est bien, bon et fidèle serviteur ; parce que tu as été fidèle en peu de choses, je t'établirai sur beaucoup ; entre dans la joie de ton maître.
}

\titre{Credo III}
\traduction{Je crois en un seul Dieu le Père tout-puissant, créateur du ciel et de la terre, de toutes choses, visibles et invisibles. Je crois en un seul Seigneur Jésus-Christ, le Fils unique de Dieu, né du Père avant tous les siècles : Dieu né de Dieu, lumière née de la lumière, vrai Dieu né du vrai Dieu, engendré non pas créé, consubstantiel au Père, et par qui tout a été créé. C'est lui qui, pour nous, les hommes, et pour notre salut, est descendu des cieux ; \emph{(Ici, on fait la genuflexion)} il a pris chair de la Vierge Marie par l'action du Saint-Esprit et il s'est fait homme. Puis il fut crucifié pour nous sous Ponce-Pilate : il souffrit sa passion et fut mis au tombeau. Il ressuscita le troisième jour, suivant les Ecritures ; il monta aux cieux où il siège à la droite du Père. De nouveau il viendra dans la gloire pour juger les vivants et les morts, et son règne n'aura pas de fin.Je crois en l'Esprit-Saint, qui est Seigneur et qui donne la vie, qui procède du Père et du Fils. Avec le Père et le Fils il reçoit même adoration et même gloire. Il a parlé par les prophètes. Je crois à l'Église une, sainte, catholique et apostolique. Je reconnais un seul baptême pour la rémission des péchés et j'attends la résurrection des morts et la vie du monde à venir. Ainsi soit-il.}
\smallskip
\partition{Kyriale}{Credo-III-auto}{5.}

\begin{center}\titreb{Offertoire}
\end{center}
\traduire{\dominus}{\leseigneur}
\traduire{Orémus}{Prions}

\titre{Antienne d'offertoire}
\traduction{J'ai trouvé David mon serviteur ; je l'ai oint de mon huile sainte ; car ma main l'assistera et mon bras le fortifiera.}
\partition{Offertoire}{InveniDavid}{}

\texte{Offertoire}
\traduire{Sancti tui, qu{\'\ae}sumus, Dómine, nos ubíque lætíficant : ut, dum eórum mérita recólimus, patrocínia sentiámus.}{Que le souvenir de vos Saints nous soit, ô Seigneur, en tous lieux, un sujet de joie, afin que nous ressentions la protection de ceux dont nous célébrons à nouveau les mérites.}

\titre{%
Préface}

\rubrique{%
  Par le chant de la Préface, le Célébrant rend grâces à Dieu au nom de l'Église pour l'{\oe}uvre du salut réalisée par Jésus-Christ.}
\introprefacetraduite
\rubrique{Le prêtre chante alors la préface}
\titre{%
Sanctus}
\traduction{Saint, saint, saint le Seigneur, Dieu des Forces célestes , le ciel et la terre sont remplis de votre gloire. Hosanna au plus haut des cieux. Béni soit celui qui vient au nom du Seigneur. Hosanna au plus haut des cieux.}
\partition{Kyriale}{Sanctus-IV}{8.}

\begin{center}
\titreb{Canon}
\end{center}
\texte{Canon}
\begin{center}
\titreb{Communion}
\end{center}
\texte{Communion}
\partition{Kyriale}{Agnus-IV}{6.}
\medskip

\texte{Communionb}
\traduction{Voici le dispensateur fidèle et prudent que le Maître a établi sur ses serviteurs pour leur donner au temps fixé, leur mesure de blé.}
\partition{Communion}{FidelisServusSineAlleluia}{}

\traduire{\dominus}{\leseigneur}
\traduire{Orémus}{Prions}
\traduire{Præsta, qu{\'\ae}sumus, omnípotens Deus : ut, de percéptis munéribus grátias exhibéntes, intercedénte beáto \emph{Anthelmo} Confessóre tuo atque Pontífice, benefícia potióra sumámus. Per Dóminum.}{Accordez-nous, s'il vous plaît, ô Dieu tout-puissant, qu'en rendant grâces pour les dons reçus, nous recevions plus de bienfaits encore grâce à l'intercession du bienheureux Anthelme votre Confesseur et Pontife.}
\traduire{\dominus}{\leseigneur}
\traduction {Allez, la Messe est dite}
\partition{Kyriale}{Ite-IV}{1.}
\texte{FinMesse}


\tableofcontents
%\titre{Adoro Te}
%\bigskip
%\input{partitions/AdoroTe}{}
%
%\pagebreak
%
%%\bigskip
%\titre{Ecce Panis Angelorum}
%\partition{partitions}{EccePanis}{}
%
%\clearpage
%\titre{Ave verum}
%\partition{partitions}{AveVerum}{}
%
%
%\newpage
%\titre{\selectfont\gara{Pour le Sacré-C{\oe}ur}}
%\titre{Litanies}
%\input{partitions/litanies2.tex}
%
%%\newpage
%\titre{\selectfont\gara{Pour la très sainte Vierge}}
%
%\titre{Alma}
%\rubrique{De l'Avent à la Purification.}
%\partition{partitions}{AlmaRedemtporis}{5.}
%
%\rubrique{De l'Avent à la Nativité}
%\traduire{\textbf{\V Angelus Dómini nuntiávit Maríæ.}\\ \R Et concépit de Spíritu Sancto.}{\V L'Ange du Seigneur annonça à Marie.\\ \R Et elle conçut par le Saint-Esprit.}
%\traduire{\\\textbf{Orémus.}\\Deus, qui de beátæ Maríæ
%Vírginis útero Verbum tuum, Angelo
%nuntiánte, carnem suscípere voluísti,
%præsta supplícibus tuis * ut, qui vere
%eam Genetrícem Dei crédimus, eius
%apud te intercessiónibus adiuvémur.
%Per eúndem Christum Dóminum nostrum.}{\\ Prions\\Daignez, Seigneur, répandre votre grâce dans nos âmes, afin qu'ayant connu par la voix de l'Ange, l'Incarnation du Christ votre Fils, nous soyons conduits par sa passion et sa croix, à la gloire de la résurrection. Par le même Christ notre Seigneur.}
%\traduire{ \R Amen}{\R Ainsi soit-il}
%%\pagebreak
%\rubrique{De la Nativité à la Purification}
%\traduire{\textbf{\V Post partum, Virgo invioláta permansísti.}\\ \R Dei Génetrix intercéde pro nobis.}{\V Vous êtes demeurée sans tache après l'enfantement, ô Vierge.\\ \R Mère de Dieu, intercédez pour nous.}
%\traduire{\textbf{Orémus.}\\Deus, qui salútis ætérnæ,
%beátæ Maríæ virginitáte fecúnda,
%humáno géneri pr{\'\ae}mia præstitísti: 
%tríbue qu{\'\ae}sumus, ut ipsam pro nobis 
%intercédere sentiámus, * per quam 
%merúimus auctórem vitæ suscípere, 
%Dóminum Jesum Christum Fílium tuum}{Prions.\\O Dieu, qui par la virginité féconde de la bienheureuse Marie, avez procuré au genre humain les avantages du salut éternel ; accordez-nous, s'il vous plaît, de ressentir les effets de l'intercession de celle par qui nous avons eu la grâce de recevoir l'auteur de la vie, notre Seigneur Jésus-Christ, votre Fils.}
%\traduire{ \R Amen}{\R Ainsi soit-il}
%
%\titre{Ave Regina c\ae lorum}
%\rubrique{De la Purification à Pâques}
%\partition{partitions}{AveReginaCaelorumSimple}{6.}
%
%
%\traduire{\textbf{\V Dignáre me laudáre te Virgo sacráta.}\\ \R Da mihi virtútem contra hostes tuos.}{\V Rendez-moi digne de vous louer, ô Vierge sainte. \\ \R Donnez-moi de la force contre vos ennemis.}
%\traduire{\textbf{Orémus.}\\Concéde, miséricors Deus, 
%fragilitáti nostræ præsídium, ut, 
%qui sanctæ Dei Genitrícis me mó riam 
%ágimus, intercessiónis eius auxílio, 
%a nostris iniquitátibus resurgámus. 
%Per Christum Dóminum nostrum.}{Prions.\\
%Accordez, ô Dieu de miséricorde, votre secours à notre fragilité ; afin que nous, qui célébrons la mémoire de la sainte Mère de Dieu, nous puissions, à l'aide de son intercession, nous relever de nos iniquités. Par le même Christ, notre Seigneur}
%\traduire{ \R Amen}{\R Ainsi soit-il}
%%\pagebreak
%\titre{Regina c\ae li}
%\rubrique{De Pâques à la Trinité}
%\partition{partitions}{ReginaCaeliSimple}{6.}
%\vspace{0.5cm}
%\traduire{\textbf{\V Gaude et lætáre Virgo María, allelúia.}\\ \R Quia surréxit Dóminus vere, allelúia.}{\R Réjouissez-vous et soyez dans l'allégresse, Vierge Marie, alléluia.\\ \V Parce que le Seigneur est vraiment ressuscité, alléluia}
%\traduire{\textbf{Orémus.}\\Deus, qui per resurrectiónem Fílii tui Dómini nostri Jesu 
%Christi mundum lætificáre dignátus
%es:  præsta, qu{\'\ae} sumus, ut per 
%eius Genitrícem Vírginem Maríam 
%perpétuæ capiámus gáudia vitæ. 
%Per eúmdem Christum Dóminum 
%nostrum.}{Prions.\\O Dieu, qui avez daigné réjouir le monde par la résurrection de Jésus-Christ, votre Fils ; faites, nous vous en supplions, qu'aidés par sa Mère, la Vierge Marie, nous arrivions à la possession des joies de la vie éternelle. Par le même Christ notre Seigneur.}
%\traduire{ \R Amen}{\R Ainsi soit-il}
%
%\titre{Salve Regina}
%\rubrique{De la Trinité à l'Avent}
%\partition{partitions}{SalveReginaSimple}{1.}
%\traduire{\textbf{\V Ora pro nóbis sancta Dei Génetrix.}\\ \R Ut digni effi ciámur promissiónibus Christi.}{\V Priez pour nous, sainte Mère de Dieu.\\ \R Afin que nous devenions dignes des promesses de Jésus-Christ.}
%\traduire{\textbf{Orémus.}\\Omnípotens sempitérne 
%Deus, qui gloriósæ Vírginis Matris Maríæ dignum Fílii tui habitáculum éffici mererétur, Spíritu Sancto præparásti da, ut cuius commemoratióne lætámur  eius pia intercessióne ab instántibus malis et a morte perpétua liberémur. 
%Per eúmdem Christum Dóminum 
%nostrum.}{Prions.\\Dieu tout-puissant et éternel, qui, par la coopération du Saint-Esprit, avez préparé le corps et l'âme de la glorieuse Marie Vierge Mère, pour qu'elle méritât de devenir la digne demeure de votre Fils, faites que celle dont nous nous réjouissons de célébrer la mémoire, nous délivre, par sa miséricordieuse intercession, des maux présents et de la mort éternelle. Par le même Christ notre Seigneur.}
%\traduire{ \R Amen}{\R Ainsi soit-il}
%
%\newpage
%\titre{Inviolata}
%\rubrique{Pour l'année}
%\partition{partitions}{Inviolata}{4.}
%
%\newpage
%\titre{\selectfont\gara{Pour le Pape}}
%\titre{Tu es Petrus}
%\partition{partitions}{TuEsPetrus}{7.}
%\medskip
%\titre{Oremus pro pontifice nostro}
%\partition{partitions}{Oremus}{7.}
%\footnotetext[1]{On ne descend sur le mi qu'à la dernière syllabe du nom du Pape}
%%\clearpage
%\bigskip
%\traduire{\textbf{\V Tu es Petrus.}\\ \R Et super hanc petram ædificábo\linebreak Ecclésiam meam.}{\R Tu es Pierre\\ \V Et sur cette pierre je bâtirais mon Église}
%\rubrique{ou}
%\traduire{\textbf{\V Fiat manus tua super virum dexteræ tuæ.}\\ \R Et super filium hominis quem confirmasti tibi.}{\V Que votre main soit sur l'homme de votre droite. \\ \R Et sur le fils de l'homme que vous avez choisi.}
%\traduire{\textbf{Orémus.}\\Deus ómnium fidélium pastor et rector, fámulum tuum N., quem 
%pastórem Ecclésiæ tuæ pr{\'\ae}esse voluísti, propítius réspice; da ei qu{\'\ae} sumus, 
%verbo et exémplo quibus pr{\'\ae}est, profícere; ut ad vitam, una cum grege 
%sibi crédito, pervéniat sempitérnam. Per Christum Dóminum nostrum.}{Prions.\\Dieu éternel et tout-puissant, ayez pitié de votre serviteur notre Pape N., et dans votre bonté, guidez-le dans la voie du salut éternel, en sorte que par le don de votre grâce, il recherche ce qu'il vous plaît, et l'accomplisse de tout son pouvoir, par le Christ notre Seigneur}
%\traduire{ \R Amen}{\R Ainsi soit-il}
%
%%\newpage
%\titre{\selectfont\gara{Avant la bénédiction}}
%
%\titre{Tantum ergo I}
%\partition{partitions}{Tantum1}{3.}
%\medskip
%\traduire{\V Panem de c\ae lo præstitisti eis (\textit{T.P. et Corpus Christi :} alleluia)}{\V Vous leur avez donné le pain du ciel,  (\textit{T.P. et Fête-Dieu :} alleluia}
%\traduire{\textbf{\R Omne delectaméntum in se habéntem (alleluia).}}{\R Qui renferme toutes sortes de délices (alleluia).}
%\traduire{\textbf{Orémus.}\\Omnípotens sempitérne 
%Deus qui nobis sub Sacraménto mirábili Passionis tuæ memoriam reliquisti tribue, qu\' \ae sumus, ita nos corporis et sanguinis tui sacra mystéria venerári ; ut redempti\' onis tuæ fructum in nobis júgiter sentiámus: Qui vivis et regnas in sæcula sæculorum.}
%{Prions.\\O Dieu, qui nous avez laissé un souvenir continuel de votre passion dans le Sacrement admirable de l'Eucharistie, faites-nous la grâce de révérer de telle sorte les mystères sacrés de votre corps et de votre sang, que nous ressentions sans cesse en nous le fruit de votre rédemption. Vous qui, étant Dieu, vivez et régnez avec Dieu le Père en l'unité du Saint-Esprit, dans tous les siècles des siècles.}
%\traduire{ \R Amen}{\R Ainsi soit-il}
%
%\titre{\selectfont\gara{Reposition du saint Sacrement}}
%\rubrique{On entonne le chant de reposition quand le prêtre monte pour reposer la lunule}
%\titre{Cor Jesu sacratissimum}
%\partition{partitions}{CorJesu}{1.}
%\vspace{0.5cm}
%
%\begin{parcolumns}[nofirstindent]{2}
%\colchunk[2]{
%\parbox[t][][t]{6cm}{
%\titre{C\oe ur sacré de Jésus}
%\medskip
%C\oe ur sacré de Jésus,\\Que votre règne arrive,\\
%C\oe ur sacré de Jésus,\\ Je crois en votre amour pour moi,\\
%C\oe ur sacré de Jésus,\\ J'ai confiance en vous\\
%}
%}
%\colchunk[1]{
%\parbox[t][][t]{6cm}{
%\titre{Jésus doux et humble de c\oe ur}
%\medskip
%\textit{Refrain}
%\\
%\textbf{Jésus, Jésus, doux et humble de c\oe ur}
%\vspace*{-0.2cm}
%\[
%\left.
%\begin{array}{ll}
%\hspace*{-1.4cm}\mbox{1. Rendez mon  c\oe ur \textit{(bis)}}\\
%\hspace*{-1.4cm}\mbox{Semblable au vôtre.}
%\end{array}
%\right\}bis
%\]
%2. Placez mon c\oe ur...\\
%Bien près du vôtre.
%
%\medskip
%3.Prenez mon c\oe ur...\\
%Qu'il soit bien vôtre.
%
%\medskip
%4.Brûlez mon c\oe ur...\\
%Au feu du vôtre.
%
%\medskip
%5.Changez mon c\oe ur...\\
%Avec le vôtre.
%
%
%}
%
%}
%\end{parcolumns}
%
%%\clearpage
%\titre{Règne à jamais}
%%\medskip
%
%\begin{parcolumns}[nofirstindent]{2}
%\colchunk{\noindent
%\noindent
% {1.  Règne à jamais, C\oe ur glorieux,\\
%Dans tous les temps, dans tous les lieux,\\
%Sur terre, comme dans les cieux. \\
%\\
%\textbf{Ô C\oe ur sacré, sois notre Roi !\\
%Nous voulons vivre sous ta loi !\\
%Nous n'aimerons jamais que Toi !\\}
%\\
%2.  Règne à jamais sur nos foyers;\\
%Sur eux toujours reviens veiller :\\
%Avec foi nous saurons prier.\\}}
%
%\colchunk{\noindent
%\noindent
%{3. Aux peuples tremblants dans leur foi,\\
%Il faut un chef, il faut un Roi !\\
%Ce Roi sauveur, Jésus c'est Toi !\\
%\\
%4. Depuis qu'à Reims au temps jadis,\\
%Tu baptisas le fier Clovis,\\
%Tu dois régner sur nous, ses fils.\\
%\\
%5. Règne, ô Jésus, sur tous les c\oe urs,\\
%Sur tes amis, sur les pécheurs,\\
%Sur les brebis, sur les pasteurs. \\}}
%
%\end{parcolumns}
%%\newpage
%%\thispagestyle{empty}
%\vspace*{6.5cm}
%\begin{center}
%\includegraphics[height=3.5cm]{images/SC.pdf}
%\end{center}

\end{document}